%\documentclass[final]{siamltex}
\documentclass[a4paper]{article}
\usepackage{amsmath,amssymb}
\DeclareMathOperator{\Diag}{diag}
\usepackage{algorithm,algpseudocode,algorithmicx}  % For pseudocode algorithms
\usepackage{listings,url,verbatim}    % Could we do without listings?
 
%\usepackage{subfigure}
%\usepackage{longtable}
%\usepackage{lscape}
%\usepackage{multirow}
\usepackage[pdftex]{graphicx}
\usepackage{rotating}
\usepackage{color}
\usepackage{graphics}

\newcommand{\lt}{\left}
\newcommand{\rt}{\right}

\newcommand\abs[1]{\lvert#1\rvert}
\newcommand\norm[1]{\lVert#1\rVert}


\title{Project $4$: On the generation of Random Orthogonal matrices and Random SVD}

\author{Amal Khabou, Lijing Lin }
%\thanks{ School of Mathematics, The University of
%    Manchester, Manchester, M13 9PL, England.} ({\tt
%    amal.khabou@manchester.ac.uk}).
%\and Nick Higham \thanks{School of Mathematics, The University of
%    Manchester, Manchester, M13 9PL, England.} ({\tt
%    nhigham@manchester.ac.uk}). }

\begin{document} 

\maketitle

\section{The problem to solve} 

Random matrices is a topic that illustrates the deep and strong connections between numerical analysis and statistics 
\cite{} (book in the email from Nick).

For testing purposes we often need to form ``random SVD matrices'', i.e., matrices of the form $A = UDV^T$ , 
where $U$ and $V$ are random orthogonal matrices and $D$ is diagonal with a 
particular distribution of singular value on the diagonal.


The real problem is how to compute $QB$, where $B$ is a given matrix and $Q$ 
is a random orthogonal matrix. This is what MATLAB's gallery('qmult') does. 
Once this problem is solved, the random SVD problem is an immediate corollary.
qmult($B$) returns $QB$ where $Q$ is a random real orthogonal matrix from the 
Haar distribution of dimension the number of rows in $B$. 
qmult($B$, method), if the method is set to 1 it means that the computations 
use a call to the QR factorization. This is implies more floating-point operations, 
however numerical experiments show that this method is much faster for large dimensions.\\

\textbf{Goal:} We aim at developing an efficient algorithm that only performs 
$O(kn^2)$ floating point operations to generate random orthogonal matrices, instead of $O(n^3)$.

%And indeed this is the problem that most often arises. For example, qmult is widely used for generating random matrices with particular structures \cite{Betcke:20%12:NCN}, \cite[Alg. 1]{nh_jortho}, and there are plenty of other applications of it.

\section{Generating randsvd matrices}
A randsvd matrix $A$ is given by $A = U \Sigma V^T$, where $U$ and $V$ are random orthogonal matrices and 
$ \Sigma = diag(\sigma_i), (\sigma_i \geq 0)$ is formed by given singular values. Thus the problem of 
generating randsvd matrices is reduced to the generation of random orthogonal matrices. In the following 
we recall the two main methods in literature used to generate random orthogonal matrices.\\

An ensemble of random matrices is defined by a matrix space and a probability measure on it.
The set of $n \times n$ orthogonal matrices is a compact topoligical group.

\paragraph{The Haar distribution \\}

add a section for approximatelly Haar

Some numerical applications require generation of uniformly distributed random orthogonal matrices. For 
example, the Haar measure is essential to analyze the statistical behavior of the eigenvalues and the eigenvectors 
of complex sample covariance matrices. Here the uniformity is 
defined in terms of Haar measure, that means that the distribution is still unchanged if multiplied by any given orthogonal 
matrix.  \\

\textbf{Definition 1:}
A measure $\mu(E)$ is a generalized volume defined on $E$. A measure $\mu$,
defined on a group $G$, is a Haar measure if $\mu(gE) = \mu(E)$, for every $g \in G$.
For the group $O(n)$ of orthogonal $n \times n$ matrices, the condition for this
measure to be a Haar is that, for any continuous function $f$,\\

$\int_{Q \in O(n)} f(Q) d_\mu(Q) = \int_{Q \in O(n)} f(Q_1Q) d_\mu(Q)$, for any $Q_1 \in  O(n)$.\\

In other words, if $Q$ is a set of orthogonal matrices with probability
$\mu$, then the sets $Q_1Q$ and $Q Q_1$ have the same probability $\mu$ for any $Q_1 \in  O(n)$.\\



\textbf{Definition 2:}
The probability measure $h_n$ defined on the Borel $\sigma$-field $\mathbb{B}_{on}$ of Borel subsets of 
$\mathbb{O}_n$ is called Haar measure if, for any Borel set $A \in \mathbb{B}_{on}$ and orthogonal matrix 
$\textbf{O} \in \mathbb{O}_{n}$ , $h_n(\textbf{O} A) = h_n(A)$, where $\textbf{O} A$ denotes the set of all 
$\textbf{O} \textbf{A} $, $\textbf{A}  \in A$.\\

The proofs of the following properties are detailed in \cite[Chapter 10]{bai2009spectral}.\\

\textbf{Property 1:}
If $\textbf{H}_n$ is $h_n$-distributed, then for any unit vector $x_n$, $y_n = \textbf{H}_nx_n$ 
is uniformly distributed on the unit $n$-sphere.\\

\textbf{Property 2:}
If $\textbf{H}_n$ is $h_n$-distributed, then $\textbf{H}'_n$ is also $h_n$-distributed.\\

\textbf{Property 3:}
If $\textbf{Z}$ is a $n\times n$ matrix with entries iid $N(0,1)$, then $\textbf{U} = \textbf{Z}(\textbf{Z}′\textbf{Z})^{−1/2}$ 
and $\textbf{V} = (\textbf{Z}\textbf{Z}')^{−1/2}\textbf{Z}$  are $h_n$-distributed.\\
We note here that $\textbf{U}$ is the polar factor of $\textbf{Z}$.\\

\textbf{Property 4:}
Assume that on a common probability space, for each $n$, $\textbf{H}_n$  is $h_n$-distributed and $\textbf{x}_n$ is 
a unit vector. Let $\textbf{y}_n = (y_1,\cdots , y_n)′ = \textbf{H}'_n\textbf{x}_n$ and $f$ a bounded continuous function. 
Then, as $ n \to \infty$, 
\begin{center}
${{1}\over{n}} \sum_{j=1}^{n}f(\sqrt{n}y_j) \to \int f(x)\varphi(x)dx$, almost surely, \\
\end{center}
where $\varphi(x)$ is the density of $N(0,1)$.\\

\textbf{Property 5:}

The eigenvalues of a random orthogonal matrix, \{$e^{\theta_1},\ldots,e^{\theta_n}$\}, lie on the unit circle.
A classical calculation in random matrix theory is the computation of the statistical correlations 
between the arguments $\theta_i$ \cite{mehta2004random}. In \cite{ANU:298726} Mezzadri consider a simple 
correlation function which consists of the density of the eigenvalues $\rho(\theta)$. He also 
considers the spacing distribution which is the probability density of the normalized distance 
between consecutive eigenvalues: \\

 $s_j = {N \over {2\pi}} ({\theta_{j+1}} - {\theta_j}), j = 1, \ldots, N.$ \\

In \cite{DiaconisSh94} the authors show that the trace of a random orthogonal matrix has an approximate 
Gaussian distribution.

\textbf{Asymptotic Haar:}

\subsection{Using Householder transformations \\}



\paragraph{Via standard QR factorization \\}

Basically using Matlab notations one should start by generating an $n \times n$ random matrix, $A = randn(A)$ 
then performs its QR factorization to get a random orthogonal matrix $Q$. 
Here we present similar experiments to those presented in \cite[section 4.6]{ANU:298726} to get 
an intuition on whether the random orthogonal matrix generated follows the Haar distribution or not. \\

The QR factorization as implemented in Matalab and elsewhere does not guarantee that the diagonal 
entries of the $R$ factor are nonnegative. Thus the obtained $Q$ factor is not guaranteed to be a Haar distributed 
orthogonal matrix.

\begin{figure}[!htb]
%\centering
\includegraphics[scale=.7]{matlabQR.eps}
\caption{The data are computed from the eigenvalues of ten thousand 
$50\times 50$ random orthogonal matrices output of Matlab QR function.}
\label{fig:0}
\end{figure}

\paragraph{The Stewart's method \\}
A Householder transformation is a symmetric orthogonal matrix of the form,\\
$H = I - {{uu^T} \over { \norm{u}_2^2}}$.
The standard algorithm used by LAPACK and MATLAB's gallery('randsvd') is based on an algorithm of 
Stewart \cite{stewart1980efficient} for generating random orthogonal matrices from the Haar distribution. 
It constructs $Q$ as a product of random Householder transformations. Basically the randsvd routine calls the 
qmult method which pre-multiplies a given matrix $A$ (the matrix of singular values in this particular case) with $Q$ 
a random real orthogonal matrix from the Haar distribution. We note that the qmult method does not form the matrix $Q$ 
explicitly. To generate a random orthogonal matrix one can apply the qmult method to the identity matrix.
Below is a description of the algorithm proposed by Stewart and implemented in the qmult function:\\
for $i = 2$ to $n$
\begin{itemize}
\item generate a vector $u$ with $i$ entries, each element is an independent normal $(0; 1)$
random number.
\item let $x$ be an $n$-vector, where the $n-i$ first entries are zero and the rest is set to $u$.
\item let $H$ be an $n \times n$ Householder transformation which reduces $x$ to a vector where
the only nonzero element is the $(n-i+1)$-th element.
\item apply the Householder transformation $H$ to $A$ : $A =  A H$
\end{itemize}
endfor\\
for $i = 1$ to $n$
\begin{itemize}
\item multiply the $i$-th column of $A$ by an independent random number $1$ or $-1$ with probability $1/2$ (is uniformly distributed on the unit circle)
\end{itemize}
endfor \\

This method is implemented by the $G05PXF$ routine in NAG Library. \\

\begin{figure}[!htb]
%\centering
\includegraphics[scale=.7]{qmult.eps}
\caption{The data are computed from the eigenvalues of ten thousand 
$50\times 50$ random orthogonal matrices output of qmult function.}
\label{fig:qmult}
\end{figure}

Fran'idea: we start from a diagonal matrix $D$ where the eigenvalues are uniformally distributed 
on the unit circle. Then we apply successive Householder transformations to $D$.
Basically we generate a matrix \\
 $Q = H_{n-1}\ldots H_k \ldots H_1 D H_1 \ldots H_k \ldots H_{n-1}$, 
where  \\
$D  = diag
 \begin{pmatrix}
  \begin{pmatrix}
  a_i & b_i  \\
  -b_i & a_i 
 \end{pmatrix}
 \end{pmatrix}
$


\textbf{Comparison:}
The following figures show the density of the phases of $10000$ matrices 
of size $50 \times 50$ using an increasing number of Householder transformations. 

\begin{itemize}

\item Product of Householder transformations

\begin{figure}[!htb]
\centering
\includegraphics[scale=.7]{qmult_1.eps}
\caption{One Householder transformation.}
\label{fig:1}
\end{figure}

\begin{figure}[!htb]
\centering
\includegraphics[scale=.7]{qmult_2.eps}
\caption{Two Householder transformation.}
\label{fig:2}
\end{figure}

\begin{figure}[!htb]
\centering
\includegraphics[scale=.7]{qmult_3.eps}
\caption{Three Householder transformation.}
\label{fig:3}
\end{figure}

\begin{figure}[!htb]
\centering
\includegraphics[scale=.7]{qmult_25.eps}
\caption{$25$ Householder transformation.}
\label{fig:25}
\end{figure}

\begin{figure}[!htb]
\centering
\includegraphics[scale=.7]{qmult_35.eps}
\caption{$35$ Householder transformation.}
\label{fig:35}
\end{figure}

\begin{figure}[!htb]
\centering
\includegraphics[scale=.7]{qmult_40.eps}
\caption{$40$ Householder transformation.}
\label{fig:40}
\end{figure}

\begin{figure}[!htb]
\centering
\includegraphics[scale=.7]{qmult_45.eps}
\caption{$45$ Householder transformation.}
\label{fig:45}
\end{figure}

\begin{figure}[!htb]
\centering
\includegraphics[scale=.7]{qmult_45.eps}
\caption{$45$ Householder transformation.}
\label{fig:45}
\end{figure}

\begin{figure}[!htb]
\centering
\includegraphics[scale=.7]{qmult_49.eps}
\caption{$49$ Householder transformation.}
\label{fig:49}
\end{figure}

We can see from the previous figures that when we decrease the number of applied Householder 
transformations, the density the angle $0$ increases, it means that the multiplicity of real 
eigenvalues increases, in particular $1$ and $-1$ since the matrix is orthogonal and all the 
eigenvalues lie in the unit circle. According to some numerical experiments, it is the multiplicity 
of $1$ which increases when the number of Householder transformation decreases.
We note that in all the figures, the blue line represents a theoretical uniform density of the 
angels $\theta$, thus $\rho(\theta) =  {1 \over {2 \pi}}$. 
If we consider the application of $k$ Householder transformations $H_1 \ldots H_k$, we note 
$Q_k = H_1H_2 \ldots H_k$ their product. The question is basically what is the distribution 
of $Q_k$ and how close it is to be Haar distributed. If $Q_k$ is explicitly required then 
the process cost is $O(kn^2)$. 

\begin{figure}[!htb]
\centering
\includegraphics[scale=.7]{eig1multiplicity.eps}
\caption{The multiplicity of $1$ for a $20 \times 20$ matrix.}
\label{fig:multiplicity}
\end{figure}


\item Householder transformations applied to the diagonal matrix $D$ defined in the previous section


\begin{figure}[!htb]
%\centering
\includegraphics[scale=.9]{Diag_H1.eps}
\caption{The data are computed from the eigenvalues of ten thousand 
$50\times 50$ random orthogonal matrices applying one Householder transformation to $D$.}
\label{fig:qmult}
\end{figure}

\begin{figure}[!htb]
%\centering
\includegraphics[scale=.9]{Diag_H5.eps}
\caption{The data are computed from the eigenvalues of ten thousand 
$50\times 50$ random orthogonal matrices applying $5$ Householder transformations to $D$.}
\label{fig:qmult}
\end{figure}

\begin{figure}[!htb]
%\centering
\includegraphics[scale=.9]{Diag_H48.eps}
\caption{The data are computed from the eigenvalues of ten thousand 
$50\times 50$ random orthogonal matrices applying $48$ ($n-2$) Householder transformations to $D$.}
\label{fig:qmult}
\end{figure}

\end{itemize}

\paragraph{Using Givens rotations (Anderson et al's method)\cite{doi:10.1137/0908055} \\}
The Anderson, Olkin and Underhill algorithm constructs the random orthogonal matrix $Q$ as 
a product of Givens rotations.\\

$Q = G_{1,2}G_{1,3} \ldots G_{1,n}G_{2,3} \ldots G_{2,n} \ldots G_{n-1,n} D,$\\

where $D$ is a diagonal matrix with random entries and $G_{i,j}$ Givens rotations.
This algorithm has the same time complexity as the Stewart's method. The main difficulty 
of this method is the generation of the angles.\\

{\color{red}To Do: finish the Matlab code for (Anderson et al's method) using the hints provided in \cite{Genz_methodsfor}.}
\paragraph{Comparison \\}
 

Note that the same process can be applied to generate a random symmetric matrix $A = Q \Lambda Q^T$ with a given eigenvalue distribution.

$Q$ here is an orthogonal Hessenberg matrix.

\section{Some ideas from LAPACK working note 9 \cite{lawn09}}
The method of interest consist on the generation of either a real symmetric or complex Hermitian 
matrix with given eigenvalues and bandwidth, or a random non symmetric or complex symmetric matrix
with given singular values and upper and lower bandwidth. The simplest way is described in details 
in \cite[section 3]{lawn09} (xLATMS routine). Basically starting from a random matrix $A$, its diagonal $D$ 
is set to the desired eigenvalues or singular values. The obtained matrix is first pre and post-multiplied 
by random orthogonal matrices. Then Householder transformations are applied to eliminate nonzero entries 
outside the specified bandwidth. The cost of this algorithm is $O(n^3)$, which could be improved to $O(n^2k)$ 
using Givens rotations, where $k$ is the specified bandwidth. This method is described in details in
 \cite[section 9.2]{lawn09} in the case of the generation of an upper triangular band matrix.\\

{\color{red}To Do: I should look at how the random orthogonal matrices used in the xLATMS routine are 
generated when Householder transformations are used to generate a random band matrix. \\}

{\color{red}To Do: I need to see to which extent it is possible to apply some techniques from 
\cite{Ballard:2012:CAS:2370036.2145822} (Avoiding Communication in successive Band Reduction) to 
efficiently generate random band matrices with specified eigenvalues or singular values.}

\section{Investigation of the ideas proposed by Nick}

\subsection{Using Cayley transform}
The Cayley transform is a mapping between skew-symmetric matrices and special orthogonal matrices.
To generate an orthogonal matrix $Q$, we consider a skew-symmetric matrix $A$, that is $A^T = −A$. 
Then $I + A$ is invertible, and the Cayley transform $Q = (I - A)(I + A)^{-1}$ is an orthogonal matrix.
The question to answer is how to choose $S$ and which distribution should it have to make $Q$ an 
orthogonal matrix from the Haar distribution.

To illustrate the density of the angles corresponding to the eigenvalues of a sample of $10000$ matrices 
of size $50$, we generate skew symmetric matrices as follows, $A = randn(n)$ and $B = A - A^T$ is the 
skew symmetric matrix used in the Caley transform.

\begin{figure}[!htb]
\centering
\includegraphics[scale=.7]{caley_50_randn.eps}
\caption{Empirical histograms of the density of the eigenvalues using Caley transform.}
\label{fig:caley1}
\end{figure}

In figure \ref{fig:caley2}, the skew symmetric matrix $B$ is generated using the following Matlab 
code, $c =randn(n-1,1);$ $d = zeros(n,1);$ $B = gallery('tridiag',c,d,-c)$.

\begin{figure}[!htb]
\centering
\includegraphics[scale=.7]{caley_50_tridiag.eps}
\caption{Empirical histograms of the density of the eigenvalues using Caley transform from tridiag matrix.}
\label{fig:caley2}
\end{figure}

\subsection{Taking selected rows from an orthogonal matrix-Using sparse embedding}
The abstract of the talk suggested by Nick: Sketching methods could be summarized as follows: a sketch of a 
matrix $A$ is its product $SA$ with a sketching matrix $S$, where $S$ is chosen so that $SA$ behaves likes A, 
in the sense that for all vectors $x$, $\norm{SAx}$ is about the same as $\norm{Ax}$. From this *subspace 
embedding* property follows a number of applications where $SA$ can be used in place of $A$, and yield provably 
good appropriate solutions. When $S$ has a small number of rows, so does the sketch $SA$, and using $SA$ in 
place of $A$ results in fast approximation algorithms.\\

How could this be applied in the context of random orthogonal matrices??

\subsection{Using the exponential of a skew symmetric matrix}

\begin{figure}[!htb]
\centering
\includegraphics[scale=.7]{expm_50_randn.eps}
\caption{Empirical histograms of the density of the eigenvalues using matrix exponential.}
\label{fig:expm1}
\end{figure}

\begin{figure}[!htb]
\centering
\includegraphics[scale=.7]{expm_50_tridiag.eps}
\caption{Empirical histograms of the density of the eigenvalues using matrix exponential (tridiag).}
\label{fig:expm2}
\end{figure}

\section{Some applications which might be interesting}
Random orthogonal matrices are used to randomize integration methods for n-dimensional 
integrals over spherically symmetric integration regions \cite{Genz_methodsfor}.
In Stewart paper, one of the applications is the estimation of the condition number of a general matrix.  
\subsection{The generation of correlation matrices}
The algorithms introduced by Higham and Davies in \cite{Davies00numericallystable} present a numerically stable way to 
generate correlation matrices with a given spectrum. The difference with previous implementations \cite{lin_Bendel85,FNAG} 
consists on the way the angles used for the Givens rotations are computed. However the most expensive step 
for those algorithms is still the generation of random orthogonal matrices, which the authors assume that 
they should be Haar distributed and hence use Stewart algorithm to generate them. We note also that 
there is no attention given to the statistical distribution of the generated matrices.
So the main question is again how to efficiently generate a random orthogonal matrix. 

\subsection{Random orthogonal matrix simulation}

ROM simulation method is a fast matrix method for generating random samples using 
random orthogonal matrices. It has applications to any problem where historical 
simulation is commonly applied.



\bibliographystyle{siam}
{
\bibliography{randsvd}
}

\end{document} 


